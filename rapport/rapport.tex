\documentclass{article}
\usepackage[utf8]{inputenc}

\usepackage{graphicx}
\usepackage{color}
\usepackage{float}
\usepackage[pdf]{graphviz}
\usepackage{amsmath}
\usepackage{amssymb}

\usepackage{geometry}
\geometry{hmargin=2.5cm,vmargin=1.5cm}

\begin{document}

\begin{figure}[t]
\centering
\includegraphics[width=5cm]{inp_n7.png}
\end{figure}

\title{\vspace{4cm} \textbf{Equations aux dérivées partielles}}
\author{El Bouzekraoui Younes | MDAA Saad}
\date{\vspace{7cm} Département Sciences du Numérique - Deuxième année \\
2020-2021 }

\maketitle

\newpage
\tableofcontents

\newpage
%%%%%%%%%%%%%%%%%%%%%%%%%%%%%%%%%%%%%%%%%%%%%%%%%%%%%%%%%%%%%%%%%%
\section{Equations aux dérivées partielles elliptiques}
\subsection{Partie théorique}
$\bullet$ On suppose que $u \in H^{1}(\Omega)$ \\
pour $w \in H_{0}^{1}(\Omega)$ on a :
$$
- \int_{\Omega} \triangle u . w \,dx = \int_{\Omega} f . w \,dx
$$
d'apres la formule de green :
$$
\int_{\Omega} \nabla u . \nabla w \,dx - \int_{\partial \Omega} \gamma_{1}(u) \gamma_{0}(w) \,dx = \int_{\Omega} f . w \,dx
$$
on pose $v = u - u_d \in H_{0}^{1}(\Omega)$ on a:
$$
\int_{\Omega} \nabla (v + u_d) . \nabla w \,dx  = \int_{\Omega} f w \,dx + \int_{\partial \Omega_{d}} \gamma_{1}(u) \gamma_{0}(w) \,dx + \int_{\partial \Omega_{n}} \gamma_{1}(u) \gamma_{0}(w) \,dx 
$$
or 
$$
\gamma_1(u) = \frac{\partial u}{\partial n} = g \textrm{ sur } \partial \Omega_{n}
$$
et
$$
\gamma_0(w) = w = 0 \textrm{ sur } \partial \Omega_{d} \textrm{ car } w \in H_{0}^{1}(\Omega)
$$
donc 
$$
\boxed{\int_{\Omega} \nabla v . \nabla w \,dx = \int_{\Omega} f w \,dx + \int_{\partial \Omega_{n}} g \gamma_{0}(w) \,dx - \int_{\Omega} \nabla u_d . \nabla w \,dx}
$$
\\
$\bullet$ Soit la \textbf{forme bilinéaire} sur $H_{0}^{1}(\Omega) \times H_{0}^{1}(\Omega)$ définie par 
$$
a : (u, v) \rightarrow \int_{\Omega} \nabla u . \nabla v \,dx
$$
soit la \textbf{forme} sur $H_{0}^{1}(\Omega)$ définie par
$$
l : w \rightarrow \int_{\Omega} f w \,dx + \int_{\partial \Omega_{n}} g \gamma_{0}(w) \,dx - \int_{\Omega} \nabla u_d . \nabla w \,dx
$$
en appliquant le théoreme de Lax-Milgram sur $a$ et $l$ on a 
$$
\boxed{\exists ! u \in H_{0}^{1}(\Omega) \; \; \forall v \in H_{0}^{1}(\Omega) \; \; a(u, v) = l(v)}
$$
pour cela il faut vérifier que \\
$\rightarrow$ \textbf{$l$ est linéaire} : \\
soient $w_1$, $w_2$ $\in H_{0}^{1}(\Omega)$, $\lambda \in \mathbb{R}$ on a 
$$
l(w_1 + \lambda w_2) = \int_{\Omega} f (w_1 + \lambda w_2) \,dx + \int_{\partial \Omega_{n}} g \gamma_{0}(w_1 + \lambda w_2) \,dx - \int_{\Omega} \nabla u_d . \nabla (w_1 + \lambda w_2) \,dx
$$
$$
= \int_{\Omega} f w_1 \,dx + \int_{\partial \Omega_{n}} g \gamma_{0}(w_1) \,dx - \int_{\Omega} \nabla u_d . \nabla w_1 \,dx + \lambda(\int_{\Omega} f w_2 \,dx + \int_{\partial \Omega_{n}} g \gamma_{0}(w_2) \,dx - \int_{\Omega} \nabla u_d . \nabla w_2 \,dx)
$$
$$
l(w_1 + \lambda w_2) = l(w_1) + \lambda l(w_2)
$$
$\rightarrow$ \textbf{$l$ est continue} : \\
soit $w$ $\in H_{0}^{1}(\Omega)$ on a 
$$
|l(w)| = |\int_{\Omega} f w \,dx + \int_{\partial \Omega_{n}} g \gamma_{0}(w) \,dx - \int_{\Omega} \nabla u_d . \nabla w \,dx|
$$
$$
\leq |\int_{\Omega} f w \,dx| + |\int_{\partial \Omega_{n}} g \gamma_{0}(w) \,dx| + |\int_{\Omega} \nabla u_d . \nabla w \,dx|
$$
par inégalité de cauchy schwarz car $f \in L^2(\Omega)$ et $w \in H_{0}^{1}(\Omega) \subset L^2(\Omega)$ on a
$$
|\int_{\Omega} f w \,dx| \leq ||f||_{L^2(\Omega)} ||w||_{L^2(\Omega)} 
$$
par inégalité de Poincaré on a $\exists C_1 \geq 0$ tq
$$
|\int_{\Omega} f w \,dx| \leq C_1 ||f||_{L^2(\Omega)} |w|_{1, \Omega} 
$$
$g \in L^2(\partial \Omega_n)$ et $\gamma_{0}(w) \in L^2(\partial \Omega_n)$ donc par inégalité de cauchy schwarz on a
$$
|\int_{\partial \Omega_{n}} g \gamma_{0}(w) \,dx| \leq ||g||_{L^2(\partial \Omega_n)} ||\gamma_{0}(w)||_{L^2(\partial \Omega_n)}
$$
par inégalité de Poincaré on a $\exists C_2 \geq 0$ tq
$$
|\int_{\partial \Omega_{n}} g \gamma_{0}(w) \,dx| \leq C_2 ||g||_{L^2(\partial \Omega_n)} |w|_{1, \Omega}
$$
$w$ $\in H_{0}^{1}(\Omega)$ et $u_d$ $\in H_{0}^{1}(\Omega)$ donc par inégalité de cauchy schwarz on a (car $\Omega$ est un ouvert borné)
$$
|\int_{\Omega} \nabla u_d . \nabla w \,dx| \leq |u_d|_{1, \Omega} |w|_{1, \Omega}
$$
donc 
$$
|l(w)| \leq C |w|_{1, \Omega}
$$
d'ou la continuité \\ \\
$\rightarrow$ \textbf{$a$ est continue} : \\
par continuité du produit scalaire \\ \\
$\rightarrow$ \textbf{$a$ est coercive} : \\
soit $u \in H_{0}^{1}(\Omega)$ on a par inégalité de cauchy schwarz
$$
|a(u, u)| = |<u, u>_{1, \Omega}| \leq |u|^{2}_{1, \Omega}
$$
\\
$\bullet$ on note $n$ le nombre de degrés de liberté et $\eta_k$ les fonctions de base des éléments finis définie par 
$$
\forall k \in [1, n] \; \; \eta_k(x_j, y_j) = \delta_{k, j} 
$$
on cherche une solution dans l'espace engendré par $(\eta_{k})_{[1, n]}$ donc $v$ s'ecrit comme 
$$
v = \sum_{k=1}^{n} v_k n_k
$$
on injecte dans l'équation de la question 1 avec $w = \eta_j \; \; j \in [1, n]$ et 
$$
u_d = \sum_{k=1}^{n} U_k \eta_k
$$
avec 
$$
U_k = \left\{
    \begin{array}{ll}
        0 & \mbox{si } (x_k, y_k) \notin \partial \Omega_{n} \\
        u_d(x_k, y_k) & \mbox{sinon.}
    \end{array}
\right.
$$
$$
\int_{\Omega} \nabla (\sum_{k=1}^{n} v_k \eta_k) . \nabla \eta_j \,dx = \int_{\Omega} f \eta_j \,dx + \int_{\partial \Omega_{n}} g \gamma_{0}(\eta_j) \,dx - \int_{\Omega} \nabla (\sum_{k=1}^{n} U_k \eta_k) . \nabla \eta_j \,dx
$$
donc 
$$
\sum_{k=1}^{n} v_k (\int_{\Omega} \nabla \eta_k . \nabla \eta_j \,dx) = \int_{\Omega} f \eta_j \,dx + \int_{\partial \Omega_{n}} g \gamma_{0}(\eta_j) \,dx - \sum_{k=1}^{n} U_k (\int_{\Omega} \nabla \eta_k . \nabla \eta_j \,dx)
$$
ceci pour $j \in [1, n]$ ce qui est équivalent au système linéaire d équations $\boxed{Ax = b}$ avec 
$$
A_{kj} = \int_{\Omega} \nabla \eta_k . \nabla \eta_j \,dx
$$

$$
b_j = \int_{\Omega} f \eta_j \,dx + \int_{\partial \Omega_{n}} g \eta_j \,dx - \sum_{k=1}^{n} U_k (\int_{\Omega} \nabla \eta_k . \nabla \eta_j \,dx)
$$
$\rightarrow$ existence de la solution \\ \\
$A$ est définie postive donc le système admet une unique solution en effet pour $x \in \mathbb{R}^n$
$$
x^{T}Ax = \sum\nolimits_{(i, j) \in [1, n]^2} A_{ij} x_i x_j = \sum\nolimits_{(i, j) \in [1, n]^2} a(\eta_i, \eta_j) x_i x_j
$$
par bilinéairité de $a$ on a
$$
x^{T}Ax = a(\sum_{j=1}^{n} x_j \eta_j, \sum_{i=1}^{n} x_i \eta_i) = a(z, z)
$$
avec $z = \sum_{i=1}^{n} x_i \eta_i$ or $a$ et continue donc 
$$
x^{T}Ax \geq C |z|^{2}_{(1, \Omega)}
$$
or si $|z|_{(1, \Omega)} = 0$ donc $z = 0$ alors $x=0$
\end{document}